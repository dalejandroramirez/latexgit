\nclase{2}
\section{Nociones básicas del flujo de trabajo}

Los estados de nuestros archivos los podemos clasificar en uno de los siguentes 
tres grupos: 

\vspace{5 mm}
\begin{description}
 \item[Working tree] (árbol de trabajo) Es el área de trabajo actual donde reposan los documentos 
 están siendo modificados. Cuando creamos o modificamos un archivo, estos 
 cambios no serán salvados en git hasta que no pasen a la staging area o al 
 repositorio local.
 
 Pasamos archivos de Working tree al staging área cada vez que usamos el comando
 \codigo{git add}
 seguido del nombre del archivo.
 
\vspace{5 mm}
 \item[Staging area] (área de preparación) En esta área se encuentran los archivos salvados por git,
  en el estado que fueron salvados. A estos archivos git les está haciendo 
  seguimiento y están a la espera de un commit para pasar al repositorio local. 
  Se se editan estos archivos, los cambios solo se verán reflejados en el 
  working tree. 
 
  Cuando corremos el comando
  \codigo{git status}
  git nos muestra a cuales archivos se encuentran en el working tree y por lo 
  tanto no se les está haciendo seguimiento (untracked); git además nos muestra 
  los archivos que estan en el staging área pendientes de un commit para pasar 
  al repositorio local.
  
\vspace{5 mm}
 \item[Local Repository] (repositorio local)
  El repositorio local se encuentran todos los puntos de control y los 
  comentarios, es el área donde se guarda todo.
  
  Solo se recomienda hacer hacer commits para pasar al repositorio local cuando
  engloben partes completas, como una funcionalidad nueva, arreglo de un error o
  añadir una nueva función. No se suelen introducir errores con los commit;
  Si he añadido una nueva función al código, probar antes que funciona
  correctamente. Todos los commit deben tener un mensaje claro y breve del 
  contenido de los cambios.
  
  Podemos saltarnos el paso por el staging área utilizando la bandera -a en el 
  commando git commit
  
 \codigo{ git commit -a}
 
 de esta forma pasaran al repositorio local todos los archivos del working tree
  o del stagin area que se les ha estado dando seguimiento. 
  
\end{description}

\newpage 

\section{Iteracciones con git}

\begin{description}
 \item[head] es el alias con el que se representa el punto de control (snapshot)
 actual del proyecto.
 
 \item[git log -p] Nos da más detalles de las diferencias entre snapshots.
 
 \item[git show ID] Usa el ID de un commit como parámetro y muestra información
 del commit y su patch asociado.
 
 \item[git log --stat] da otra información sobre los cambios, como la cantidad
 de líneas agregadas o eliminadas. 

 \item[git diff --staged] Sirve para  ver lo que has preparado y será incluido 
 en la próxima confirmación. Este comando compara tus cambios preparados con la 
 última instantánea confirmada.
 
 \item[git rm] Para eliminar ficheros.
 
 \item[git mv] para cambial el nombre de los ficheros
 
 \item[echo xxx $>$ .gitignore] para ignorar el archivo xxx.
 
 \item[git add *] para agregar todo lo del working tree al staging area
\end{description}


 
